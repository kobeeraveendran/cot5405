\documentclass[12pt]{article}
\usepackage[utf8]{inputenc}
\usepackage[legalpaper, portrait, margin=0.8in]{geometry}
\usepackage{amssymb}
\usepackage{algorithm}
\usepackage{algpseudocode}

\title{Homework 1 -- COT5405}
\author{Kobee Raveendran}
\date{August 2021}

\begin{document}

\maketitle

\begin{enumerate}
    \item note: all logs are implicitly base 2, but other bases may be specified explicitly. Also, I use the Master theorem 
    template for some questions, so it is listed below.
    
    $T(n) = aT(\frac{n}{b}) + f(n)$, where $f(n) \in O(n^k\log^pn)$
    
    \begin{enumerate}
        \item $T(n) = 9T(\frac{n}{2})+n^3 \longrightarrow O(n^{\log9}) \longrightarrow O(n^{2\log3})$ \\
        
        Steps: 
        
        using the Master theorem, we have:
        
        $a = 9$
        
        $b = 2$
        
        $f(n) = n^3 \longrightarrow k = 3, p = 0$
        
        Since $\log_{2}9 > 3$, we follow case 1 of the Master theorem ($\log_{b}a > k$), which means the bound is in 
        $O(n^{\log_{b}a})$. After substitution, that gives us:
        
        $O(n^{\log9}) = O(n^{2\log3})$\\
        
        \item $T(n) = 7T(\frac{n}{2}) + n^3 \longrightarrow O(n^3)$ \\
        
        Steps:
        
        using the Master theorem, we have:
        
        $a = 7$
        
        $b = 2$
        
        $f(n) = n^3 \longrightarrow k = 3, p = 0$
        
        We find that this recurrence falls under case 3 of the theorem, in which $\log_{b}{a} < k$ 
        (since $(\log_{2}{7} < 3)$), so we arrive at a complexity of the form $O(n^k) \longrightarrow O(n^3)$.\\
        
        \item $T(n) = T(\sqrt{n}) + \log n$
        
        Steps:

        using iterative substitution:

        $k = 0$: $T(n) = T(n^\frac{1}{2}) + \log n$

        $k = 1$: $T(n^\frac{1}{2}) = T(n^\frac{1}{4}) + \log n^\frac{1}{2}$\\

        Substituting back in, we have:

        $T(n) = T(n^\frac{1}{4}) + \log n^\frac{1}{2} + \log n = T(n^\frac{1}{4}) + \frac{1}{2}\log n + \log n$\\

        Which can be generalized to:

        $T(n) = T(n^\frac{1}{2^{k + 1}}) + (1 + \frac{1}{2} + \frac{1}{4} + .. + \frac{1}{2^k}) \log n \approx T(n^\frac{1}{2^{k + 1}}) + 2\log n$\\

        Using a base case of $T(2) = c$, we can solve for $k$:

        $n^\frac{1}{2^{k+1}} = 2 \longrightarrow \log n^\frac{1}{2^{k+1}} = \log2 = 1 \longrightarrow$
        $\frac{1}{2^{k + 1}}\log n = 1 \longrightarrow \log n = 2^{k + 1} \longrightarrow$ \\
        $\log\log n = \log 2^{k + 1} \longrightarrow \log\log n = k + 1 \longrightarrow k = \log\log n - 1$\\

        Substituting $k$ back in, as well as $T(2) = c$, we have:
        
        $T(n^\frac{1}{2^{\log\log n}}) + 2\log n \longrightarrow c + 2\log n$\\

        Giving us a time complexity of $O(\log n)$, which can be confirmed using the Master theorem by substituting 
        $n = 2^m$ to get a compatible recurrence format.

        \item $T(n) = \sqrt{n}T(\sqrt{n}) + n \longrightarrow O(n\log\log n)$
        
        Steps:

        using iterative substitution, we have:

        $T(n) = n^\frac{1}{2}T(n^\frac{1}{2}) + n$\\

        $k = 1$: $T(n^\frac{1}{2}) = n^\frac{1}{4}T(n^\frac{1}{4}) + n^\frac{1}{2}$
        
        Substituting: $T(n) = n^\frac{1}{2}[n^\frac{1}{4}T(n^\frac{1}{4}) + n^\frac{1}{2}] + n$
        $\longrightarrow T(n) = n^{\frac{1}{2} + \frac{1}{4}}T(n^\frac{1}{4}) + n + n$\\

        $k = 2$: $T(n^\frac{1}{4}) = n^\frac{1}{8}T(n^\frac{1}{8}) + n^\frac{1}{4}$

        Substituting: $T(n) = n^{\frac{1}{2} + \frac{1}{4} + \frac{1}{8}}T(n^\frac{1}{8}) + n^{\frac{1}{2} + \frac{1}{4} + \frac{1}{8}} + n + n$\\

        Simplifiying the fractional sums (which converge to 1), we have the general form:

        $T(n) = nT(n^\frac{1}{2^{k + 1}}) + (k + 1)n$\\

        Using base case $T(2) = c$ and solving for $k$, we have:

        $n^\frac{1}{2^{k + 1}} = 2 \longrightarrow \frac{1}{2^{k + 1}}\log n = 1 \longrightarrow \log\log n = \log2^{k + 1}$
        $\longrightarrow \log\log n - 1 = k$\\

        Substituting $k$ back in and using the base case, we have:

        $cn + (\log\log n)n \longrightarrow O(n\log\log n)$\\
        
        \item $T(n) = 3T(\frac{n}{3}) + \frac{n}{3}$ \\
        
        Steps:
        
        using the Master theorem, we have:
        
        $a = 3$
        
        $b = 3$
        
        $f(n) = \frac{n}{3} \longrightarrow k = 1, p = 0$
        
        This recurrence falls under case 2 of the theorem, in which $\log_{b}{a} = k$ (since $\log_{3}{3} = 1$), so our 
        complexity is of the form $O(n^k \log n) \longrightarrow O(n\log_{3}{n})$ (base changes since we divide the work 
        done each recurrent step by 3 rather than the usual 2).\\
        
        \item $T(n) = T(\frac{n}{2}) + T(\frac{n}{4}) + n\log n$
        
        Steps:

        can't directly apply the Master theorem, but we can split the RHS since we can assume that $T(\frac{n}{4}) \leq T(\frac{n}{2})$ 
        (since less of the input is processed in the recursive call where the input is quartered rather than  halved), 
        leaving us with the following true inequality:

        \centerline{$2T(\frac{n}{4}) + n\log n \leq T(n) \leq 2T(\frac{n}{2}) + n\log n$}

        Finding the asymptotic bounds of either side lets us narrow down the bounds for $T(n)$:\\

        (using the Master theorem for both)

        LHS:

        $a = 2$

        $b = 4$

        $f(n) = n\log n \longrightarrow k = 1, p = 1$

        $\log_{b}{a} = \log_{4}{2} = \frac{1}{2} \longrightarrow \frac{1}{2} < 1 \longrightarrow$ case 3 of Master theorem
        $\longrightarrow O(n\log n)$\\

        RHS:

        $a = 2$

        $b = 2$

        $f(n) = n\log n \longrightarrow k = 1, p = 1$

        $\log_{b}{a} = \log_{2}{2} = 1 = k \longrightarrow$ case 2 of Master theorem 
        $\longrightarrow O(n\log n)$\\

        Since $T(n)$ is thus bounded on both sides by $O(n\log n)$, it is safe to claim that $T(n) \in O(n\log n)$.

    \end{enumerate}
    \item blah
    \item dynamic programming ($O(n)$):
    overall idea is to store maximum partial sums that we've encountered as we traverse through the array. For each index, 
    we'd store the max partial sum if the ending index, $k$, were to be at that index, and also track the starting index, 
    $j$, of each max partial sum's subarray. At each index, we will decide whether to ``continue'' the subarray if the 
    current element's addition will increase the sum, or start a new subarray if the current element is already greater 
    than whatever sum we have (i.e. the current sum was negative). If the current max partial sum is ever greater than 
    the overall max partial sum, we update it and also update the start and end points of the subarray, $final_j$ and $final_k$.

    \begin{algorithm}
        \caption{Dynamic programming approach ($O(n)$ time with constant space)}
        \begin{algorithmic}
            \State $j, k, final_j, final_k \gets 0$
            \State $max_{total} \gets -\infty$
            \State $max_{curr} \gets 0$
            
            \State $N \gets length(A)$
            
            \For {$i \gets 0$ to $N$}
                \If{$A[i] > max_{curr} + A[i]$}
                    \State $j \gets i$
                    \State $k \gets i + 1$
                    \State $max_{curr} \gets A[i]$
                \Else
                    \State $k \gets i$
                    \State $max_{curr} \gets max_{curr} + A[i]$
                \If{$max_{curr} > max_{total}$}
                    \State $final_j \gets j$
                    \State $final_k \gets k$
                    \State $max_{total} \gets max_{curr}$
                    \EndIf
                \EndIf
            \EndFor
        \end{algorithmic}
    \end{algorithm}
    
    \item blah
    \item idea: since the rows and columns are sorted, we can eliminate entire sections of the grid if we find an element 
    is greater than or lesser than the target to be found. In the case where an element of the matrix is greater than the 
    target, we can eliminate all rows that come after that element, and all columns that come after that element, since 
    all of those elements will also be greater than the target. The inverse applies if the current element is less than 
    the target; we'd instead eliminate all rows and columns before it, since the target can't be there. If the loop ends 
    without finding the target, the target does not exist in the matrix.
    
    View algorithm 2 below
    
    \begin{algorithm}
        \caption{Iterative elimination approach ($O(n)$ time with constant space)}\label{alg:gridsearch}
        \begin{algorithmic}
            \State $N \gets length(A)$
            \State $i \gets 0$
            \State $j \gets N - 1$
            
            \While{$i < N$ and $j \leq 0$}
                \If{\texttt{target} $= A[i][j]$}
                    \State \texttt{return} $(i, j)$
                \Else
                    \If{\texttt{target} $> A[i][j]$}
                        \State $j \gets j - 1$
                    \Else
                        \State $i \gets i + 1$
                    \EndIf
                \EndIf
            \EndWhile
        \end{algorithmic}
    \end{algorithm}
    
    \item use something like quicksort's partitioning algorithm?
\end{enumerate}

\end{document}
